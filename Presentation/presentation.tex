\documentclass[xcolor=dvipsnames]{beamer}
\usepackage[utf8]{inputenc}
\usepackage{lmodern}
\usepackage[T1]{fontenc}
\usepackage{bigfoot} % to allow verbatim in footnote
\usepackage[numbered,framed]{matlab-prettifier}

\newsavebox{\firstexample}
\newsavebox{\secondexample}
\setbeamercolor{frametitle}{fg=White}
\setbeamercolor{title}{fg=White}
\usetheme{lankton-keynote}
\hypersetup{pdfstartview={Fit}}
\let\ph\mlplaceholder % shorter macro
\lstMakeShortInline"
\lstset{
  style              = Matlab-editor,
  basicstyle         = \mlttfamily,
  escapechar         = ",
  mlshowsectionrules = true,
}

\author{Christopher Lillthors Viktor Kronvall}
\title{Pilbågen}

\begin{document}
\maketitle
\setbeamertemplate{frametitle}[default][center]

%define MATLAB-code boxes.
\begin{lrbox}{\firstexample}
\lstinputlisting[caption = {findroot.m}]{../Labb3/findroot.m}
\end{lrbox}

\begin{lrbox}{\secondexample}
\lstinputlisting[caption = {Sample code from Matlab}]{../Labb3/findroot.m}
\end{lrbox}

\begin{frame}
\frametitle{Problem}
%include the first box here.
\usebox{\firstexample}
%Present the problem here...
\end{frame}
\begin{frame}
\frametitle{Solution}
%Present a working solution here...
\framesubtitle{How we worked out an solution}
%More content goes here
\end{frame}
\end{document}